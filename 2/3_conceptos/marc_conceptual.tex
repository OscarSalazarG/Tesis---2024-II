\subsection{Plataformas Digitales de Contenido Breve}
Las plataformas digitales han transformado el consumo de información mediante videos cortos que priorizan la rapidez y concisión. Este formato se ha vuelto esencial para captar la atención de usuarios que demandan información rápida y directa.

\subsection{Sobrecarga Informativa}
La constante producción de contenido audiovisual genera un exceso de información, dificultando la identificación de los momentos más relevantes. La sobrecarga informativa requiere métodos eficaces para procesar grandes volúmenes de datos y extraer lo esencial.

\subsection{Automatización de Procesos}
La automatización permite reducir la intervención humana en tareas repetitivas como la selección y edición de contenido. Esto optimiza la curaduría de videos, mejorando la eficiencia operativa y la precisión en la selección de material significativo.

\subsection{Curaduría de Contenido}
La curaduría consiste en seleccionar y organizar de manera eficiente la información más relevante, especialmente en medios informativos. La curaduría automatizada facilita la creación de resúmenes informativos a partir de contenido audiovisual extenso.

\subsection{Big Data}
Big Data implica el análisis de grandes volúmenes de datos, incluyendo videos, para extraer patrones significativos. En el contexto de tu investigación, Big Data es fundamental para manejar, procesar y analizar el contenido audiovisual generado continuamente.

\subsection{Resúmenes Informativos}
Los resúmenes informativos son versiones condensadas de videos extensos, que destacan los puntos clave de una noticia o evento. Son esenciales para que los usuarios accedan rápidamente a la información más relevante.

\subsection{Metadatos}
Los metadatos proporcionan información adicional sobre el contenido de los videos, como etiquetas, descripciones y estadísticas de visualización. Estos datos son útiles para la organización, búsqueda y análisis de videos noticiosos.

\subsection{Eficiencia Operativa}
La eficiencia operativa es la capacidad de optimizar recursos y tiempo en la producción de contenido. La automatización y el uso de tecnologías avanzadas son esenciales para lograr una mayor eficiencia en la curaduría y edición de videos.

\subsection{Atención del Usuario}
El tiempo de atención de los usuarios ha disminuido, lo que genera la necesidad de adaptar los formatos de contenido. Los medios deben crear videos breves y atractivos para mantener la atención de su audiencia y fomentar una mayor interacción.

\subsection{Selección de Contenido Relevante}
La selección de contenido relevante es el proceso de identificar los momentos más significativos dentro de un video extenso. La automatización en esta tarea permite una curaduría más rápida y precisa, adaptada a las demandas del público.
