\subsection{Scraping}
Scraping es el proceso de extracción automática de información desde páginas web. Este método permite recopilar grandes volúmenes de datos de manera eficiente y organizada, especialmente útil en contextos donde se necesita obtener y estructurar datos no disponibles de forma directa en una API o base de datos. Para su implementación, se utilizan herramientas y librerías, como \texttt{BeautifulSoup} y \texttt{Selenium}, que facilitan la navegación y extracción de elementos específicos dentro de una página web, como textos, imágenes o enlaces. El scraping es particularmente relevante en el análisis de datos y la curaduría de contenido, ya que permite alimentar sistemas de análisis y machine learning con datos en tiempo real o en grandes cantidades.

\subsection{Automatización}
La automatización es el proceso de emplear tecnología para realizar tareas de forma autónoma, con mínima intervención humana. En el ámbito de procesamiento de datos y análisis de contenido, la automatización facilita la ejecución de procesos repetitivos y complejos, como la extracción y organización de grandes volúmenes de datos, mejorando tanto la precisión como la eficiencia. En el contexto de contenido audiovisual, la automatización permite aplicar algoritmos que identifican patrones y seleccionan datos relevantes sin necesidad de intervención manual en cada paso, asegurando que el sistema pueda manejar el flujo constante de información. Además, la automatización es fundamental para lograr consistencia en el tratamiento de los datos, reduciendo errores y tiempo en cada fase del proceso, desde la recolección de datos hasta la generación de resultados finales, optimizando el análisis en contextos de alta carga informativa.

\subsection{Curaduría de contenido}
La curaduría de contenido es un proceso que involucra la selección, organización y presentación de la información más relevante dentro de un amplio conjunto de datos. En el contexto de la información digital, la curaduría se convierte en una herramienta clave para filtrar y destacar elementos específicos que son de interés para el usuario. Esto es especialmente útil en medios audiovisuales, donde se busca identificar los momentos clave en grandes volúmenes de contenido para crear resúmenes informativos. La curaduría automatizada utiliza algoritmos y técnicas de machine learning para analizar y seleccionar segmentos de interés, optimizando así el consumo de información. Esta automatización de la curaduría también ayuda a personalizar la información en función de patrones específicos de uso, haciendo que la presentación del contenido sea más precisa y atractiva para audiencias específicas.

\subsection{Preprocesamiento de datos}
El preprocesamiento es una fase crítica en el análisis de datos, especialmente en el tratamiento de imágenes y videos. Involucra una serie de técnicas que mejoran la calidad de los datos y preparan el contenido visual para su posterior análisis. Entre los procesos de preprocesamiento comunes se incluyen la normalización de colores, que ajusta los valores de color para que sean coherentes en todas las imágenes; el redimensionado, que ajusta el tamaño de los frames; y la eliminación de ruido, que remueve elementos no deseados en los datos visuales. Estas técnicas permiten que los datos sean interpretados de manera uniforme por los algoritmos de aprendizaje automático, asegurando una mayor precisión y consistencia en el análisis de grandes volúmenes de información. El preprocesamiento también contribuye a reducir el tiempo de procesamiento, optimizando el flujo de trabajo en proyectos de análisis masivo.

\subsection{Extracción de características}
La extracción de características es el proceso mediante el cual se identifican y capturan los atributos visuales más representativos de una imagen o un video. Este proceso convierte el contenido visual en vectores de datos numéricos que pueden ser fácilmente manipulados y analizados por algoritmos de machine learning.

\subsection{Clustering}
El clustering es una técnica de aprendizaje no supervisado que organiza datos en grupos o clusters basados en sus similitudes. Esta técnica es particularmente útil en el análisis de contenido visual, donde el clustering permite agrupar frames de video que presentan características visuales similares, facilitando la identificación de segmentos relevantes en el contenido. El clustering ayuda a reducir la complejidad en el análisis de datos masivos, permitiendo una estructuración que optimiza la selección de frames clave y mejora la eficiencia de los resúmenes automáticos.

\subsection{Reducción de dimensionalidad}
La reducción de dimensionalidad es una técnica que simplifica la estructura de los datos, eliminando variables redundantes o irrelevantes sin perder la esencia de la información. En el contexto de análisis visual, la reducción de dimensionalidad permite optimizar el uso de los datos al extraer solo aquellas características que aportan mayor valor informativo. Esto no solo reduce el tiempo y recursos necesarios para el procesamiento, sino que también permite mejorar la precisión de los modelos de machine learning al reducir el ruido en los datos.

\subsection{Evaluación de clusters}
La evaluación de clusters es el proceso de analizar la calidad y coherencia de los grupos formados en un conjunto de datos, asegurando que los datos agrupados sean internamente homogéneos y estén bien diferenciados de otros clusters. En proyectos de análisis visual, la evaluación de clusters es crucial para garantizar que los grupos formados representen de manera precisa distintos segmentos del contenido. Esto permite optimizar la selección de frames clave y asegurar que el contenido sea adecuado para la creación de resúmenes automáticos.

\subsection{Selección de frames clave}
La selección de frames es el proceso de identificar y elegir momentos específicos o representativos de un video para su análisis posterior. Esta técnica permite reducir significativamente el número de frames que se deben procesar, optimizando el uso de recursos computacionales y mejorando la eficiencia en el análisis de contenido. La selección de frames clave se realiza en función de métricas de representatividad, permitiendo que los segmentos seleccionados reflejen de manera precisa los momentos importantes del video. En la creación de resúmenes automáticos, la selección de frames clave facilita la generación de una representación visual concisa sin perder detalles importantes.

\subsection{Metadatos}
Los metadatos son datos descriptivos que acompañan al contenido multimedia y proporcionan información adicional sobre el archivo, como su fecha de creación, duración, etiquetas y descripción. En el análisis de videos, los metadatos permiten organizar, buscar y filtrar contenido de manera estructurada, mejorando la accesibilidad y facilitando la clasificación y curaduría de contenido audiovisual. Los metadatos son fundamentales para gestionar grandes volúmenes de datos, ya que permiten una organización eficiente de la información y ofrecen contexto sobre cada archivo de video, facilitando su identificación y análisis.

\subsection{Etiquetado de datos}
El etiquetado de datos es el proceso de asignar identificadores o etiquetas a elementos de un conjunto de datos para facilitar su clasificación y análisis. En el caso de los frames de video, el etiquetado permite identificar y organizar los diferentes momentos en función de su contenido, ayudando a estructurar los datos para un análisis más eficiente.

\subsection{Concatenación de características}
El stacking de características es un proceso que combina múltiples vectores de características en una estructura unificada, permitiendo representar diversas propiedades de los datos de manera conjunta. En el análisis de contenido visual, el stacking permite integrar características obtenidas a través de diferentes métodos de extracción en una sola estructura. Esto mejora la riqueza descriptiva de los datos y optimiza la precisión de los algoritmos en tareas de clasificación y clustering, proporcionando al modelo una visión más completa de los patrones y atributos presentes en el contenido visual.


\subsection{Extracción de Frames}
La extracción de frames consiste en seleccionar imágenes de un video a intervalos específicos, lo que permite obtener una muestra representativa del contenido visual. Este proceso es crucial para análisis posteriores, ya que reduce el volumen de datos y facilita la manipulación de imágenes.

\subsection{Aplicación de Línea de Comando}
La aplicación de línea de comando permite ejecutar scripts y comandos específicos directamente en el sistema operativo. Esta técnica es útil en el procesamiento de datos y facilita la automatización de tareas como la descarga y organización de archivos en proyectos de gran escala.

\subsection{DataFrame}
El \textit{DataFrame} es una estructura de datos en formato tabular, que permite organizar y manejar grandes volúmenes de datos de manera eficiente. Cada columna puede contener diferentes tipos de datos, y esta organización es ideal para el análisis estructurado en proyectos de procesamiento de datos y machine learning.

\subsection{Concatenación de Videos}
La concatenación de videos es el proceso de unir varios segmentos de video en uno solo. Este procedimiento es particularmente útil para crear resúmenes visuales de clips relevantes, permitiendo una presentación continua de los eventos clave sin interrupciones.

